\setuppagenumbering[location={footer,middle}]
\setupwhitespace[line]
\setuphead[chapter]
\setupcaptions[spaceafter, way=bytext, prefixsegments=none]
\usemodule[units]

\starttext

\startfrontmatter

\dontleavehmode
\blank[17em]

% Title page

\startalignment[center]
  {\bfd Safetrain User Manual}
  \blank[2*medium]
  {\tfa \italic{Keeping trains safe since 2018}}
\stopalignment

\page

% Table of Contents
\completecontent % with title

\stopfrontmatter

\startbodymatter

\chapter{Introduction}

Safetrain\trademark{} provides a complete solution for automatically
managing and maintaining transportation grids. It is tailored to model
train routes but can also be used for shipping routes, commuter traffic,
and many other transportation models. This user manual contains
information for installing and interfacing with Safetrain\trademark{}.

\chapter{Installing Safetrain\trademark{}}r

Safetrain\trademark{} provides support for Windows, Macintosh, and Linux
operating systems. However, this section will focus on installing
Safetrain\trademark{} for Linux. To run it on Ubuntu or Debian, you will
need Python3 and Pip3. If you do not have othese installed run

{\tt \$ sudo apt install python3 pip3}

You will also need the Tkinter library for Python3, which can be
installed by running

{\tt \$ pip3 install tkinter}

\chapter{Running Safetrain\trademark{}}

To run Safetrain\trademark{} change directory to the directory
containing the source file {\tt main.py}. Then, providing a text file
containing train IDs and station IDs, run

{\tt \$ python3 main.py textvals.txt}

\stopbodymatter

\stoptext
